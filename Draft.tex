\documentclass{scrartcl}

\usepackage[hidelinks]{hyperref}
\usepackage[none]{hyphenat}
\usepackage{setspace}
\doublespace

\title{Different shading techniques for video games.}
\subtitle{COMP110 - Research Journal}
\author{1702208}
\date{2017-11-08}

\begin{document}

\maketitle{}

\newpage{}

\section{Introduction}
There are a lot of shading techniques nowadays.
Some of them are pretty old, but still capable to improve on.
Those are discussed in B. T. Phong's paper published in 1975 \cite{phong1975illumination}.
Although there are more newer papers that discuss several variations of shadings \cite{zhang1994analysis}.
Many of them are usable in games, which are also fast and realistic \cite{iones2003fast}.
In addition, some can be combined to produce better shading for objects \cite{blinn1976texture}.
A good shading technique in modern video games is a must nowadays.
This will be discussed later in this paper to make it more constructed, of course.
Moreover, lightings must fit the day-night cycle to make it more realistic and immersive.
Because most players look into graphics rather than gameplay, this makes the development team concentrate on realistic environment.
Thus, a lot of games nowadays are made exclusively to make players fully immerse into the given environment.
In addition with a rich story, the game becomes satisfying to play, although not for everyone.
But not about the story now, let's concentrate on the shading and lighting here, shall we?

\section{Old, but Gold}
Techniques that are described in \cite{phong1975illumination} are old, but are a good example of beginning shadings. TODO: look how to cite in this way
Some of them were not precise, but they were improved lated on.
Furthermore, human's eyes play a huge role in visualizing realistic graphics.
``Human visual perception and the fundamental laws of optics are considered in the development of a shading rule that provides better quality and increased realism in generated images."  \cite{phong1975illumination} TODO: cite appropriately
I assume this is worth mentioning, but not discussing a lot as this paper is aimed to talk about video games' graphics.
So, there are a few shadings discussed in that paper.
First of all, Warnock's shading.
Two effect were simulated: ``Decreasing intesity of the reflected light from the object with the distance between the light source and the object"
and ``Highlights created by specular reflection". TODO: appropriate citation
For more details, see \cite{phong1975illumination}
Second shading described in that paper is Newell, Newell , and Sancha's shading.
This type of shading presents transparency and highlights.
It also highlight the reflection of light.
It is not as accurate as the real object would look like, but this will later be improved.
A good sample for that is Gouraud's shading.
His technique is more advanced.
This technique improved the shading itself sufficiently.
But ``problems still exist, however, one of which is the apparent discontinuity across polygon edges." TODO: appropriate citation
After all, this has all been improved by now, and these techniques are unlikely to be used in modern games.

\section{Several different shading techniques}
I'm surely not gonna discuss all 8 of them, but some are worth mentioning, I guess.
I still haven't read all of the techniques provided in \cite{zhang1994analysis}, but I'll try to make some conclusions.


\bibliography{References}
\bibliographystyle{ieeetr}

\end{document}